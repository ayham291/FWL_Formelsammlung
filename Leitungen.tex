\section{Leitungen}

\subsection{Leitungsparameter}


\includegraphics[width=\columnwidth]{Figures/Leitungsparameter.png}

%für Leitungsparameter Grafik von Sattler füllt eine Seite
%\includegraphics[width=\columnwidth]{Figures/Leitungsparameterberechnen.png}
%\includegraphics[width=\columnwidth]{Figures/ZahlentabelleLeitungsparameter.png}
%\begin{multicols}{2}

\subsubsection{Doppelleitung:}
\[
    a = \text{Leiter Radius} \qquad d = \text{Abstand zw. den Leitern} \\
\]

\includegraphics[width=0.4\columnwidth]{Figures/Doppelleitung.png}
{\renewcommand*{\arraystretch}{0.2}
\begin{tabularx}{0.5\columnwidth}{|X|}
    \hline
    \[R  = \frac{1}{\pi a\delta\sigma_c}\]              \\
    \hline
    \[L = \frac{\mu}{\pi} \cosh^{-1}\frac{d}{2a}\]      \\
    \hline
    \[G = \frac{\pi\sigma}{\cosh^{-1}(^d/_{2a})}\]      \\
    \hline
    \[C = \frac{\pi\varepsilon}{\cosh^{-1}(^d/_{2a})}\] \\
    \hline
\end{tabularx}}

\subsubsection{Koaxial Leitung}
\[
    a = \text{innen Radius} \qquad b = \text{innen Radius Außenleiter} \\
\]

\includegraphics[width=0.4\columnwidth]{Figures/Koaxialleitung.png}
{\renewcommand*{\arraystretch}{0.2}
\begin{tabularx}{0.5\columnwidth}{|X|}
    \hline
    \[R=\frac{1}{2\pi\delta\sigma_c}\left[\frac{1}{a}+\frac{1}{b}\right]\] \\
    \hline
    \[L=\frac{\mu}{2\pi}\ln\frac{b}{a}\]                                   \\
    \hline
    \[G=\frac{2\pi\sigma}{\ln(^b/_a)}\]                                    \\
    \hline
    \[C=\frac{2\pi\varepsilon}{\ln(^b/_a)}\]                               \\
    \hline
\end{tabularx}}

\subsubsection{Parallele Platten}
\[
    w  = \text{Platten Breite} \qquad d  = \text{Abstand zw. Platten}
\]

\includegraphics[width=0.4\columnwidth]{Figures/Parallele_Platten.png}
{\renewcommand*{\arraystretch}{0.2}
\begin{tabularx}{0.5\columnwidth}{|X|}
    \hline
    \[R=\frac{2}{w\delta\sigma}\] \\
    \hline
    \[L=\frac{\mu d}{w}\]          \\
    \hline
    \[G=\frac{\sigma w}{d}\]       \\
    \hline
    \[C=\frac{w\varepsilon}{d}\]  \\
    \hline
\end{tabularx}}

\vspace{1ex}
Für beliebige Leitergeometrie gelten folgende Zusammenhänge:
\[
    LC = \mu\varepsilon \quad \text{und} \quad \frac{G}{C} = \frac{\sigma}{\varepsilon}
\]

\subsection{Allgemeine Lösung Leitungsgleichung}
\begin{align*}
    U(z) & = U^+ e^{\gamma z} + U^- e^{-\gamma z} = U^+ e^{\gamma d} + U^ - e^{-\gamma d}                      \\
    I(z) & = I^+ e^{\gamma z} + I^- e^{-\gamma z} = \frac{U^+}{Z_L}e^{\gamma d} - \frac{U^-}{Z_L}e^{-\gamma d} \\
    Z_L & = \frac{U^+}{I^+} = \sqrt{ \frac{R + j \omega L}{G + j \omega C}}
\end{align*}

\begin{align*}
    \gamma & = j \omega \sqrt{LC} \cdot \sqrt{ \frac{RG}{j^2 \omega^2 LC} + \frac{G}{j \omega C} + \frac{R}{j \omega L} + 1}\\
    \lambda & = \frac{2 \pi}{\beta} \\
    v_{Ph} & = \frac{\omega}{\beta} \\
    l_{elektr.} & = \beta \cdot l\\
    \alpha                  & = \omega \cdot \sqrt{\dfrac{\mu \varepsilon}{2}\cdot \left(\sqrt{1+\dfrac{\sigma^2}{\omega^2\cdot\varepsilon^2}}{\color{red}{-}}1\right)}   \\
    \beta                   & = \omega \cdot \sqrt{\dfrac{\mu \varepsilon}{2}\cdot \left(\sqrt{1+\dfrac{\sigma^2}{\omega^2\cdot\varepsilon^2}}{\color{green}{+}}1\right)}
\end{align*}

\subsubsection{vernachlässigbarer Wdst.belag}
\includegraphics[width=\columnwidth]{Figures/vernachlaessigbarerWiderstandsbelag.png}


\subsubsection{vernachlässigbarer Wdst.belag}
\includegraphics[width=\columnwidth]{Figures/vernachlässigbarerLeiterwertbelag.png}

\subsection{Übertragungsleitung mit Last}

\resizebox{0.4\textwidth}{!}{
    \begin{tikzpicture}
        \begin{circuitikz}%[american voltages]
            %Schaltbild
            \draw(0,0)
            to[V,v=$u_G(t)$](0,2)                               %Spannungsquelle
            to[R=$Z_g$](2,2)                                    %Quelleninnenwiderstand
            to[short,o-o](6,2)                                  %Leitung mit Knoten
            to[short](8 ,2)
            to[R= $Z_A$](8,0)                                   %Lastwiderstand
            to[short](6,0)                                      
            to[short,o-o](2,0)                                  %Leitung mit Knoten
            to[short](0,0);   
            
            %linke gestrichelte linie
            \draw[dotted](2,0)--(2,-0.5) node[left]{$l=0$};
            \draw[dotted](2,-0.5)--(2,-1) node[left]{$z=d$};
            \draw[dotted](2,-1)--(2,-1.5);

            %Pfeil in richtung l
            \draw[->](2,-0.5) -- (6,-0.5);
            \node at (3,-0.5)[above]{positiv $l$};
            
            %rechte gestrichelte Linue
            \draw[dotted](6,0)--(6,-0.5) node[right]{$l=d$};          
            \draw[dotted](6,-0.5)--(6,-1) node[right]{$z=0$};
            \draw[dotted](6,-1)--(6,-1.5);

            %pfeil in richtung z
            \draw[->](6,-1) -- (2,-1);
            \node at (5,-1)[above]{positiv $z$};

            %Pfeil in vorwärts richtung
            \draw[->](2,1.25) -- (5,1.25);
            \node at (3.5,1.25)[above]{hinlaufende Welle};

            %Pfeil in rückwärts richtung
            \draw[->](6,0.5) -- (3,0.5);
            \node at (4.5,0.5)[above]{rücklaufende Welle};
        \end{circuitikz}
    \end{tikzpicture}
}

\begin{align*}
    U(z) & = U^+ e^{\gamma z} + U^- e^{-\gamma z} = U^+ e^{\gamma d} + U^ - e^{-\gamma d}                      \\
    I(z) & = I^+ e^{\gamma z} + I^- e^{-\gamma z} = \frac{U^+}{Z_L}e^{\gamma d} - \frac{U^-}{Z_L}e^{-\gamma d}
\end{align*}

\begin{align*}
    \underline{z}_n & = \frac{\underline{Z}_A}{Z_L}                     & \underline{r} & = \frac{\underline{z}_n-1}{\underline{z}_n+1}= \frac{1-\underline{y}_n}{1+\underline{y}_n} \\
    \underline{r}_A & = \frac{\underline{Z}_A-Z_L}{\underline{Z}_A+Z_L} & m             & = \frac{1-|\underline{r}|}{1+|\underline{r}|}
\end{align*}

\subsubsection{Reflexionsfaktor entlang einer Leitung}
\begin{align*}
    r_E    & = r_A  ^{-2\gamma l} = r_A  e^{-2\alpha l} e^{-j2\beta l}                                                     \\
    \alpha & = -\frac{\ln(r_A)}{2l} [\si{Np/m}]                        & \beta & = \dfrac{\phi_2 -\phi_1}{2l} [\si{rad/m}]
\end{align*}

\subsubsection{Stehwellenverhältnis}
\begin{align*}
    \mathrm{SWR} = \frac{U_\text{max}}{U_\text{min}} =
    \frac{I_\text{max}}{I_\text{min}} = \frac{1+|r(z)|}{1-|r(z)|} =
    \frac{|U_H|+|U_R|}{|U_H|-|U_R|}
\end{align*}

\subsection{Mehrfachreflexionen bei fehlender Anpassung}

\tikz{
    %Linien
    \draw[-] (1,0) -- (1,6);
    \draw[-] (5,0) -- (5,6);

    %Pfeile mit Bezeichnungen
    \draw[->] (3.5,6.5) -- (5,6.5)node[right]{$z$};

    \draw[->] (1,6) -- (5,5) node[right]{$t_D$};
    \draw[-] (1,6) -- (3,5.5) node[above]{$U_{1h}$};

    \draw[->] (5,5) -- (1,4)node[left]{$2t_D$};
    \draw[-] (5,5) -- (3,4.5) node[above]{$U_{1r}$};

    \draw[->] (1,4) -- (5,3)node[right]{$3t_D$};
    \draw[-] (1,4) -- (3,3.5) node[above]{$U_{2h}$};

    \draw[->] (5,3) -- (1,2)node[left]{$4t_D$};
    \draw[-] (5,3) -- (3,2.5) node[above]{$U_{2r}$};

    \draw[->] (1,2) -- (5,1)node[right]{$5t_D$};
    \draw[-] (1,2) -- (3,1.5) node[above]{$U_{3h}$};

    
    \draw[dotted ] (5,1) -- (3,0.5);

    %Klammern mit Bezeichnungen
    \draw [black,
        decorate,
        decoration = {brace,
                raise=5pt,
                amplitude=5pt}] (1,4.2) --  (1,5.8);
    \draw[] (0.5,5) -- (0.5,5) node[left]{$U_{1h}$};

    \draw [black,
        decorate,
        decoration = {brace,
                raise=5pt,
                amplitude=5pt}] (5,4.8) --  (5,3.2);
    \draw[] (5.5,4) -- (5.5,4) node[right]{$U_{1h}(1+r_A)$};

    \draw [black,
        decorate,
        decoration = {brace,
                raise=5pt,
                amplitude=5pt}] (1,2.2) --  (1,3.8);
    \draw[] (0.5,3) -- (0.5,3) node[left]{$U_{1h}$};
    \draw[] (0.5,2.5) -- (0.5,2.5) node[left]{$+(1+_I)U_{1r}$};

    \draw [black,
        decorate,
        decoration = {brace,
                raise=5pt,
                amplitude=5pt}] (5,2.8) --  (5,1.2);
                \draw[] (5.5,2) -- (5.5,2) node[right]{$U_{1h}(1+r_A)$};
                \draw[] (5.5,1.5) -- (5.5,1.5) node[right]{$+U_{2h}(1+r_A)$};


    \draw [black,
        decorate,
        decoration = {brace,
                raise=5pt,
                amplitude=5pt}] (1,0.2) --  (1,1.8);
    \draw[] (0.5,1.5) -- (0.5,1.5) node[left]{$U_{1h}$};
    \draw[] (0.5,1) -- (0.5,1) node[left]{$+(1+r_I)U_{1r}$};
    \draw[] (0.5,0.5) -- (0.5,0.5) node[left]{$+(1+r_I)U_{2r}$};

}
\begin{align*}
    U_{1h} & = \frac{U_G\cdot Z_L}{R_I + Z_L}            \\
    U_{1r} & = r_A\cdot U_{1h}                           \\
    U_{2h} & = r_I\cdot U_{1r} = r_I\cdot r_A U_{1h}     \\
    U_{2r} & = r_A\cdot U_{2h} = r_I\cdot r_A^2 U_{1h}   \\
    U_{3h} & = r_I\cdot U_{2r} = r_I^2\cdot r_A^2 U_{1h}
\end{align*}
\resizebox{0.4\textwidth}{!}{
    %\centering
    \begin{tikzpicture}
        \begin{circuitikz}%[american voltages]
            \draw(0,0)
            to[V,v=$u_G(t)$](0,2)   %Spannungsquelle
            to[R=$R_I \neq Z_L$](2,2) %Quelleninnenwiderstand
            to[short](8,2)
            to[R= $R_A\neq Z_L$](8,0)
            to[short](0,0)
            (4,2) to [open, v=$u(z\mathpunct{,}t)$] (4,0) %Spannungspfeil u(z,t)
            (2,2) to [open, v=$u_E(t)$] (2,0) %Spannungspfeil u_E(t)
            (6,2) to [open, v=$u_A(t)$] (6,0); %Spannungspfeil u_A(t)
            \draw[-] (2,2) circle (0.1);
            \draw[-] (6,2) circle (0.1);
            \draw[-] (2,0) circle (0.1);
            \draw[-] (6,0) circle (0.1);
        \end{circuitikz}
    \end{tikzpicture}
}
