\section{Smith-Diagramm}

\subsection{Allgemein} \label{sec:Smith_All}


$m$             : Anpassungsfaktor

$s$             : inverser Anpassungsfaktor

$\underline{r}$ : Reflexionsfaktor

$1$             : Anpassungspunkt

\begin{center}
    \begin{align*}
        \Aboxed{r(z)  = r_A \cdot e^{-j2\beta z}}                                              \\
        \Aboxed{Z(z)  = Z_L\cdot\frac{Z_A+jZ_L\cdot\tan(\beta z)}{Z_L+jZ_A\cdot\tan(\beta z)}} \\
        \text{mit} \beta = \frac{2\pi}{\lambda}                                                \\
        \text{auch ohne Quelle gültig!}
    \end{align*}
    \input{Figures/Smithdiagramm_Smithchart.tex}
\end{center}
\begin{align*}
    \underline{z}_n & = \frac{\underline{Z}_n}{Z_L}                                                                                                                   \\
    \underline{r}_n & = \frac{\underline{Z}_n-Z_L}{\underline{Z}_n+Z_L}= \frac{\underline{z}_n-1}{\underline{z}_n+1}    = \frac{1-\underline{y}_n}{1+\underline{y}_n} \\
    m               & = \frac{1-|\underline{r}|}{1+|\underline{r}|}                                                                                                   \\
    s               & = \frac{1}{m}
\end{align*}

\subsection{Impedanz/Admetanz umrechnen}
Im Smithchart spiegeln (Phase $\pm 180^{\circ}$/$\pm \pi$)

\subsection{Zusammenschaltungen}
\begin{center}
    \includegraphics[width=.45\columnwidth]{Figures/Smithdiagramm_Zusammenschaltungen.png}
\end{center}

\columnbreak

\subsection[Von Last zu Quelle]{Lastseite $\rightarrow$ Quelle}
\begin{enumerate}
    \item $Z_L$ ins Diagramm einzeichen
    \item Lastimpedanz bestimmen,
          wenn zB Parallelschaltung etc
    \item Normieren
          \[\underline{z}_a = \frac{\underline{Z}_A}{Z_L} \]
    \item Ins Chart eintragen
    \item Linie vom Mittelpunkt durch $\underline{z}_a$ nach außen

          Ablesen und Notieren:

          $\rightarrow$Relative Länge $\left[\frac{l}{\lambda}\right]$

          $\rightarrow$Relativer Winkel
    \item Kreis einzeichen

          Ablesen und Notiere:

          $\rightarrow$Maxima: rechter Schnittpunkt mit Re-Achse

          $\rightarrow$Minima: linker  Schnittpunkt mit Re-Achse

          $\rightarrow$Rexlexionsfaktor abmessen und aus Skala oben auslesen
    \item Um Leitungslänge im UZS laufen
          $\rightarrow$ Linie vom Mittelpunkt durch neuen Punkt nach außen

          Ablesen und Notieren:

          $\rightarrow$Relativer Winkel
    \item Wenn $\alpha\neq 0$

          $\rightarrow$ Dämpung ausrechen
          $\rightarrow$ Um Faktor nach innen Spiralieren

    \item Dieser Punkt ist $\underline{z}_e$
    \item Eingangsimpedanz ablesen
          \[\underline{Z}_E = \underline{z}_e \cdot Z_L\]
\end{enumerate}