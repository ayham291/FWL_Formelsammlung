\section{Grundlagen}
\subsection{Differentialoperatoren}

\textbf{Nabla-Operator}
\begin{align*}
    \nabla & = \vec{\nabla} = 
    \begin{psmallmatrix}
        \partial  / \partial x \\
        \partial  / \partial y \\
        \partial  / \partial z
    \end{psmallmatrix}
\end{align*}

\textbf{Laplace-Operator}
\begin{align*}
    \varDelta  & = \vec{\nabla} \cdot \vec{\nabla} = \textrm{div (grad)} = 
    \dfrac{\partial ^2}{\partial x^2}+\dfrac{\partial ^2}{\partial y^2}+\dfrac{\partial ^2}{\partial z^2}
\end{align*}

\textbf{Divergenz} $\opdiv$: Vektorfeld $\rightarrow$ Skalar\\
\small{Quelldichte, gibt für jeden Punkt im Raum an, ob Feldlinien entstehen oder verschwinden. \textit{mFS S.382}}
\begin{align*}
    \opdiv \vec{F} = \nabla \cdot \vec{F}   =  \dfrac{\partial F_x}{\partial x} 
    + \dfrac{\partial F_y}{\partial y} + \dfrac{\partial F_z}{\partial z}\\ 
                                 \begin{cases}
    > 0 \quad\Rightarrow \textnormal{Quelle}  \\
    < 0 \quad\Rightarrow \textnormal{Senke} \\
    = 0 \quad\Rightarrow \textnormal{quellenfrei} 
\end{cases}                                      
\end{align*}\\
% Bsp: $\opdiv \vec{B} = 0$, da mag. Felder in sich geschlossen.\\

\textbf{Rotation} $\oprot$: Vektorfeld $\rightarrow$ Vektorfeld\\ 
\small{Wirbeldichte, gibt für jeden Punkt im Raum Betrag und Richtung der Rotationsgeschwindigkeit an. \textit{mFS S.382}}
\begin{align*}
\oprot \vec{F} = \nabla \times \vec{F} = 
\begin{pmatrix}
    \dfrac{\partial F_z}{\partial y} - \dfrac{\partial F_y}{\partial z} \\
    \dfrac{\partial F_x}{\partial z} - \dfrac{\partial F_z}{\partial x} \\
    \dfrac{\partial F_y}{\partial x} - \dfrac{\partial F_x}{\partial y}
\end{pmatrix} =
\begin{vmatrix}
    \vec{e}_x & \vec{e}_y & \vec{e}_z \\
    \dfrac{\partial}{\partial x} & \dfrac{\partial }{\partial x} & \dfrac{\partial }{\partial z} \\
    \vec{F}_x & \vec{F}_y & \vec{F}_z
\end{vmatrix}
\end{align*}\\
% Bsp: energieerhaltende (konservatives) Felder $ \rightarrow \oprot = 0$\\

\textbf{Gradient} $\opgrad$: Skalarfeld $\rightarrow$ Vektor/Gradientenfeld\\ 
\small{zeigt in Richtung steilster Anstieg von $\phi$, \textit{mFS S.380}}
\begin{align*}                                                                                          
    \opgrad \phi = \nabla \cdot \phi =  
    % \hspace{4.5ex}
    \begin{psmallmatrix}
        \partial \phi / \partial x \\
        \partial \phi / \partial y \\
        \partial \phi / \partial z
        % \dfrac{\partial G}{\partial x}\\
        % \dfrac{\partial G}{\partial y}\\
        % \dfrac{\partial G}{\partial z}
    \end{psmallmatrix}
    = \dfrac{\partial \phi}{\partial x} \vec{e}_x + \dfrac{\partial \phi}{\partial y} \vec{e}_y + 
    \dfrac{\partial \phi}{\partial z} \vec{e}_z  
\end{align*}

\subsubsection{Rechenregeln}
$\phi, \psi$: Skalarfelder \qquad $\vec{A}, \vec{B}$: Vektorfelder
\begin{align*}
     & \nabla \cdot (\vec{A} \times \vec{B}) & = & \qquad (\nabla \times \vec{A})\cdot\vec{B} - (\nabla\times\vec{B})\cdot\vec{A} \\
     & \nabla \cdot (\phi \cdot \psi)        & = & \qquad \phi (\nabla \psi) + \psi( \nabla \phi)                                  \\
     & \nabla \cdot (\phi \cdot \vec{A})           & = & \qquad \phi (\nabla \vec{A}) + \vec{A}(\nabla \phi)                             \\
     & \nabla \times (\phi \cdot \vec{A})          & = & \qquad \nabla \phi \times \vec{A} + \phi (\nabla \times \vec{A})                        
    %  & \oprot \opgrad f                      & = & \qquad 0 \Rightarrow\textnormal{Gradientenfeld Quellenfrei}                    \\
    %  & \opdiv \oprot \vec{f}                 & = & \qquad 0 \Rightarrow\textnormal{Wirbelfeld Quellenfrei}
\end{align*}

\subsubsection{Spezielle Vektorfelder}
quellenfreies Vektorfeld $\rightarrow$ Vektorpotential:
\begin{align*}
\opdiv \vec{F} = \opdiv (\oprot \vec{E}) = 0 \quad \Leftrightarrow \quad  \vec{F} = \oprot \vec{E}
\end{align*}
wirbelfreies Vektorfeld $\rightarrow$ Skalarpotential: 
\begin{align*}
    \oprot \vec{F} = \oprot (\opgrad \phi) = 0 \quad \Leftrightarrow \quad  \vec{F} = \opgrad \phi
\end{align*}
quellen- und wirbelfreies Vektorfeld:
\begin{align*}
    \oprot \vec{F} = 0 \quad \opdiv \vec{F} = 0\\
    \opdiv (\opgrad \phi) = \varDelta \phi = 0 \quad \Leftrightarrow \quad  \vec{F} = \opgrad \phi
\end{align*}

% Feldänderung bei Bewegung
% \begin{align*}
%     \Delta G & = \dfrac{\partial G}{\partial x} \Delta x + \dfrac{\partial G}{\partial y} \Delta y + \dfrac{\partial G}{\partial z} \Delta z \\
%              & = dG = \opgrad G \cdot d \vec{s}
% \end{align*}

% \columnbreak
\subsection{Logarithmische Maße}

% {\samepage
    \begin{itemize}
        \setlength\itemsep{0pt}
        \item $\si{dBm} \hat=  1\si{mW}$
        \item $\si{dB\mu} V \hat= 1\si{\mu V}$
        \item $\si{dBmV} \hat{=} 1mV$
        \item $\si{dBi} \rightarrow$ Isotropic
    \end{itemize}
    % }
    \begin{description}
        \item Dezibel [dB]
        \begin{flalign*}
            X[dB]     & = 20 \cdot \log_{} \left( \dfrac{U_1}{U_2}\right) & X[dB] & = 10 \cdot \log_{} \left( \dfrac{P_1}{P_2}\right) & \\
            U_1       & = U_2 \cdot 10^{\frac{X}{20\si{dB}}}                     & P_1   & = P_2 \cdot  10^{\frac{X}{10\si{dB}}}       & \\
            1 \si{dB} & \,\hat=                                           &       & 0,1151 \si{Np}                                    &
        \end{flalign*}
        
        \item Neper [Np]
        \begin{flalign*}
            X[Np]     & = \ln \left(\dfrac{U_1}{U_2}\right) & X[Np] & = \dfrac{1}{2} \cdot \ln \left(\dfrac{P_1}{P_2}\right) & \\
            U_1       & = U_2 \cdot e^{X}                   & P_1   & = P_2 \cdot  e^{2X}                                    & \\
            1 \si{Np} & \,\hat=                             &       & 8,686 \si{dB}                                          & % =\dfrac{20}{\ln 10} \cdot \ln \left( \dfrac{U_1}{U_2}\right)
        \end{flalign*}
    \end{description}
    
    \subsection{Vektorrechnung}
    \subsubsection{Kreuzprodukt}
    \textit{mFS S.53}
    \begin{align*}
        \vec{a}\times\vec{b} & =
            \begin{pmatrix}
                a_x \\
                a_y \\
                a_z
            \end{pmatrix}
            \times
            \begin{pmatrix}
                b_x \\
                b_y \\
                b_z
            \end{pmatrix} =
            \begin{pmatrix}
                a_yb_z-a_zb_y \\
                a_zb_x-a_xb_z \\
                a_xb_y-a_yb_x
            \end{pmatrix}
        \end{align*}

        \subsubsection{Schnittwinkel zweier Vektoren}
        \textit{mFS S.52}
        \begin{align*}
            \vec{E} \cdot \vec{H} & = |\vec{E}| \cdot |\vec{H}| \cdot cos(\varphi)                         \\
            cos(\varphi)          & = \dfrac{E_x \cdot H_x + E_y \cdot H_y + E_z \cdot H_z}{|E| \cdot |H|}
        \end{align*}
        
        \subsubsection{Betrag eines Vektors}
        \begin{align*}
        \vert \vec{r}  \vert & = r = \sqrt{r^2_x + r^2_y + r^2_z}
        \end{align*}

        \subsection{Randbedingung}
        \begin{tabularx}{0.45\textwidth}{>{\hsize=.3\hsize}X|>{\hsize=.7\hsize}X}
            Dirichlet-RB & Funktion nimmt an den Rändern einen bestimmten Wert an (Bsp.: $\rho_r = 5V$) \\
            \hline
            Neumann-RB   & Die Normalableitung der Fkt. nimmt an den Rändern einen bestimmten Wert an   \\
        \end{tabularx}

% \subsection{Begriffe}
% \begin{tabularx}{0.45\textwidth}{>{\hsize=.1\hsize}X|>{\hsize=.5\hsize}X|>{\hsize=.4\hsize}X}
%            & Begriff           & Beschreibung \\
%     \hline
%     $\rho$ & Raumladungsdichte &              \\
% \end{tabularx}




\subsection{Vergleich/Umrechnung}
\begin{tabularx}{0.45\textwidth}{>{\hsize=.46\hsize}X|>{\hsize=.27\hsize}X|>{\hsize=.27\hsize}X}
    Kart.                                                                                & Zyl.             & Kug.                            \\
    \specialrule{1.5pt}{0pt}{0pt}
    $x$                                                                                  & $r \cos \varphi$ & $r \sin \vartheta \cos \varphi$ \\
    \hline
    $y$                                                                                  & $r \sin \varphi$ & $r \sin \vartheta \sin \varphi$ \\
    \hline
    $z$                                                                                  & $z$              & $r \cos \vartheta$              \\
    \specialrule{1.5pt}{0pt}{0pt}
    $\sqrt{x^{2}+y^{2}}$                                                                 & $r$              &                                 \\
    \hline
    $\arctan \frac{y}{x}$                                                                & $\varphi$        &                                 \\
    \hline
    $z$                                                                                  & $z$              &                                 \\
    \hline
    $d x \cos \varphi+d y \sin \varphi$                                                  & $dr$             &                                 \\
    \hline
    $d y \cos \varphi-d x \sin \varphi$                                                  & $r d\varphi$     &                                 \\
    \hline
    $dz$                                                                                 & $dz$             &                                 \\
    \specialrule{1.5pt}{0pt}{0pt}
    $\sqrt{x^{2}+y^{2}+z^{2}}$                                                           &                  & $r$                             \\
    \hline
    $\arctan \frac{y}{x}$                                                                &                  & $\varphi$                       \\
    \hline
    $\arctan \frac{\sqrt{x^{2}+y^{2}}}{z}$                                               &                  & $\vartheta$                     \\
    \hline
    $d x \sin \vartheta \cos \varphi+d y \sin \vartheta \sin \varphi+d z \cos \vartheta$ &                  & $dr$                            \\
    \hline
    $d y \cos \varphi-d x \sin \varphi$                                                  &                  & $r \sin \vartheta d \varphi$    \\
    \hline
    $d x \cos \vartheta \cos \varphi+d y \cos \vartheta \sin \varphi-d z \sin \vartheta$ &                  & $r d \vartheta$                 \\
\end{tabularx}
