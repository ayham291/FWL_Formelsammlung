\section{Maxwell-Gleichungen}

\includegraphics[width=0.7\columnwidth]{Figures/Integralsatz.png}

\textbf{Amperesches- /Durchflutungsgesetz:}

\begin{tabularx}{\textwidth}{>{\hsize=.5\hsize}X>{\hsize=.5\hsize}X}
    Elek. Strom ist Ursache für ein magn. Wirbelfeld. & $\boxed{\oint_s \vec{H} \cdot d \vec{s} = \Theta = I = \iint_A \vec{J} \cdot d \vec{A} = \frac{d\Phi_e}{dt}}$ \\
\end{tabularx}

\textbf{Induktionsgesetz:}

\begin{tabularx}{\textwidth}{>{\hsize=.5\hsize}X>{\hsize=.5\hsize}X}    
    Ein sich zeitlich änderndes Magnetfeld erzeugt ein elek. Wirbelfeld. & $\boxed{\oint_s{\vec{E} \cdot d\vec{s}} = u_{ind} = -\frac{d}{dt}\iint_A{\vec{B} \cdot d\vec{A}} = -\frac{d\Phi_m}{dt}}$                          \\
                                                                               & $\boxed{rot{\vec{E}} = -\frac{\partial\vec{B}}{\partial t} = -\mu\cdot\frac{\partial\vec{H}}{\partial t} = -j\omega\mu\vec{H}}$
\end{tabularx}

\textbf{Differentielles ohmsches Gesetz:}

\begin{tabularx}{\textwidth}{>{\hsize=.5\hsize}X>{\hsize=.5\hsize}X}
    Bewegte elektrische Ladung erzeugt Magnetfeld & $\boxed{ rot \vec{H} = \vec{J} = \kappa \cdot \vec{E}} $
\end{tabularx}

Bei isotropen Stoffen sind $\varepsilon$ u. $\mu$ Skalare:
\[
    \varepsilon = \varepsilon_0 \cdot \varepsilon_r \qquad \mu = \mu_0 \cdot \mu_r
\]

Zeitbereich: $ \dfrac{\partial}{\partial t} $ \qquad \qquad 
Harmonischer Frequenzbereich (komplexe Berechnung): $ jw $

\subsection{Feldstärkekomponenten einer ebenen Welle}
Bei Ausbreitung in $z$-Richtung gibt es keine Amplitudenabhängigkeit von $x, y$ d.h. $\frac{\partial \ldots}{\partial x}=\frac{\partial \ldots}{\partial y}=0$ \\damit ergibt sich aus den Maxwell'schen Gleichungen:
{\footnotesize
	$$
	\begin{gathered}
		\boxed{\operatorname{rot} \underline{\vec{E}}=-\mathrm{j} \omega \mu \underline{\vec{H}}} \qquad \boxed{\operatorname{rot} \underline{\vec{H}}=\mathrm{j} \omega \varepsilon \underline{\vec{E}}} \\
		\operatorname{rot} \underline{\vec{E}}=\left|\begin{array}{ccc}
			\vec{e}_x & \vec{e}_y & \vec{e}_z \\
			\frac{\partial}{\partial x} & \frac{\partial}{\partial y} & \frac{\partial}{\partial z} \\
			\underline{E}_x & \underline{E}_y & \underline{E}_z
		\end{array}\right|=\left(\begin{array}{c}
			\frac{\partial \underline{E}_z}{\partial y}-\frac{\partial \underline{E}_y}{\partial z} \\
			\frac{\partial \underline{E}_x}{\partial z}-\frac{\partial \underline{E}_z}{\partial x} \\
			\frac{\partial \underline{E}_y}{\partial x}-\frac{\partial \underline{E}_x}{\partial y}
		\end{array}\right)=\left(\begin{array}{l}
			0-\frac{\partial \underline{E}_y}{\partial z} \\
			\frac{\partial \underline{E}_x}{\partial z}-0 \\
			0-0
		\end{array}\right)=-\mathrm{j} \omega \mu\left(\begin{array}{l}
			\underline{H}_x \\
			\underline{H}_y \\
			\underline{H}_z
		\end{array}\right) \\
		\operatorname{rot} \underline{\vec{H}}=\left|\begin{array}{ccc}
			\vec{e}_x & \vec{e}_y & \vec{e}_z \\
			\frac{\partial}{\partial x} & \frac{\partial}{\partial y} & \frac{\partial}{\partial z} \\
			\underline{H}_x & \underline{H}_y & \underline{H}_z
		\end{array}\right|=\left(\begin{array}{c}
			\frac{\partial \underline{H_z}}{\partial y}-\frac{\partial \underline{H}_y}{\partial z} \\
			\frac{\partial \underline{H}_x}{\partial z}-\frac{\partial \underline{H}_z}{\partial x} \\
			\frac{\partial \underline{H}_y}{\partial x}-\frac{\partial \underline{H}_x}{\partial y}
		\end{array}\right)=\left(\begin{array}{c}
			0-\frac{\partial \underline{H}_y}{\partial z} \\
			\frac{\partial \underline{H}_x}{\partial z}-0 \\
			0-0
		\end{array}\right)=\mathrm{j} \omega \varepsilon\left(\begin{array}{l}
			\underline{E}_x \\
			\underline{E}_y \\
			\underline{E}_z
		\end{array}\right)
	\end{gathered}
	$$
}

\subsection{Integralsätze}
	\begin{description}
		\setlength{\itemsep}{1pt}
		\item Fundamentalsatz der Analysis
		\item Gauß: Vektorfeld das aus Oberfläche von Volumen strömt muss aus Quelle in Volumen
		\item Stokes: innere Wirbel kompensieren sich $\rightarrow$ nur den Rand betrachten.
	\end{description}
	\begin{align*}
		\int_{a}^b \opgrad F \cdot d \vec{s}     & = F(b) - F(a)                                  \\
		\iiint_V \opdiv \vec{A} \cdot dV         & = \oiint_{ \partial V} \vec{A} \cdot d \vec{a} \\
		\iint_{A} \oprot \vec{A} \cdot d \vec{a} & = \oint_{ \partial A} \vec{A} \cdot d \vec{r}
	\end{align*}


