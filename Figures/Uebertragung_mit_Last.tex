\resizebox{0.4\textwidth}{!}{
    \begin{tikzpicture}
        \begin{circuitikz}%[american voltages]
            %Schaltbild
            \draw(0,0)
            to[V,v=$u_G(t)$](0,2)                               %Spannungsquelle
            to[R=$Z_g$](2,2)                                    %Quelleninnenwiderstand
            to[short,o-o](6,2)                                  %Leitung mit Knoten
            to[short](8 ,2)
            to[R= $Z_A$](8,0)                                   %Lastwiderstand
            to[short](6,0)                                      
            to[short,o-o](2,0)                                  %Leitung mit Knoten
            to[short](0,0);   
            
            %linke gestrichelte linie
            \draw[dotted](2,0)--(2,-0.5) node[left]{$l=0$};
            \draw[dotted](2,-0.5)--(2,-1) node[left]{$z=d$};
            \draw[dotted](2,-1)--(2,-1.5);

            %Pfeil in richtung l
            \draw[->](2,-0.5) -- (6,-0.5);
            \node at (3,-0.5)[above]{positiv $l$};
            
            %rechte gestrichelte Linue
            \draw[dotted](6,0)--(6,-0.5) node[right]{$l=d$};          
            \draw[dotted](6,-0.5)--(6,-1) node[right]{$z=0$};
            \draw[dotted](6,-1)--(6,-1.5);

            %pfeil in richtung z
            \draw[->](6,-1) -- (2,-1);
            \node at (5,-1)[above]{positiv $z$};

            %Pfeil in vorwärts richtung
            \draw[->](2,1.25) -- (5,1.25);
            \node at (3.5,1.25)[above]{hinlaufende Welle};

            %Pfeil in rückwärts richtung
            \draw[->](6,0.5) -- (3,0.5);
            \node at (4.5,0.5)[above]{rücklaufende Welle};
        \end{circuitikz}
    \end{tikzpicture}
}