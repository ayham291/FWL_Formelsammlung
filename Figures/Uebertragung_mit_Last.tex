\def\Hoehe{2};
\def\Breite{8}
\resizebox{0.4\textwidth}{!}{
    %\centering
    \begin{tikzpicture}
        \begin{circuitikz}%[american voltages]
            \draw(0,0)
            to[V,v=$u_G(t)$](0,2)                           %Spannungsquelle
            to[R=$Z_g$](2,2)                               %Quelleninnenwiderstand
            to[short](8,2)
            to[R= $Z_A$](8,0)                               %Lastwiderstand
            to[short](0,0);             
            \draw[-] (2,2) circle (0.1);                    %TOR 1 oben
            \draw[-] (6,2) circle (0.1);                    %TOR 2 oben
            \draw[-] (2,0) circle (0.1);                    %TOR 1 unten
            \draw[-] (6,0) circle (0.1);                    %TOR 2 unten
            \draw[dotted](2,0)--(2,-0.5) node[left]{$l=0$};
            \draw[dotted](2,-0.5)--(2,-1) node[left]{$z=d$};
            \draw[dotted](2,-1)--(2,-1.5);
            \draw[->](2,-0.5) -- (6,-0.5);
            \node at (5,-0,5)[above]{positiv l};

            \draw[dotted](6,0)--(6,-0.5) node[right]{$l=d$};          
            \draw[dotted](6,-0.5)--(6,-1) node[right]{$z=0$};
            \draw[dotted](6,-1)--(6,-1.5);
            \draw[->](6,-1.5) -- (2,-1.5);
            \node at (5,-1,5)[above]{positiv z};

            \draw[->](2,1.25) -- (5,1.25);
            \node at (4,1.25)[above]{forward propagating wave};
            \draw[->](6,0.5) -- (3,0.5);
            \node at (5,0.5)[above]{backward propagating wave};
        \end{circuitikz}
    \end{tikzpicture}
}